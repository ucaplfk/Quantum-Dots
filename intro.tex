\section{\label{sec:level1}Introduction}
Present day computers emerged from the principles of "the computing machine model", now referred to as the Turing machine, devised by Turing in 1936 [\citen{Turing1937OnEntscheidungsproblem,Nielsen2010QuantumInformation}]. The impactful development of the first integrated circuit (IC) in 1958 provided a cheap and fast operated basic chip element for building the computer mainboard [\citen{Arns1998TheTransistor}]. Moore's trend predictions since 1965 of the population increase by reduction in the size of transistors on a IC chip, have held accurate for the past three decades [\citen{Moore1965CramingCircuits,GordonE.Moore1975ProgressElectronics}]. However, continuing to increase processing power by the scale down method is reaching the fundamental limit of the transistor size resulting in quantum effects of electrons dominating circuit operation. Quantum computing is being investigated as an alternative speed-up method to solve problems which cannot be efficiently solved using classical computation [\citen{Waldrop2016TheLaw,Fox2006QuantumIntroduction}]. Active areas of research of many systems include: neutral atoms, trapped ions, superconducting circuits and photons assess the feasibility of realising universal quantum computation by providing the basic unit for quantum computing known as the qubit [\citen{Saffman2016QuantumChallenges,Brown2016Co-designingIons,Gambetta2017BuildingSystem,Milburn2009PhotonsQubits}].  


Quantum dots (QDs) are nanoscale structures which have potential to be implemented as a qubits due to their ability to confine and address single electrons using gate electrodes [\citen{Imamoglu2003AreComputation}]. Despite self-assembled QDs [\citen{Warburton2013SingleDots}] also being a promising candidate source of qubits, where the system is controlled optically, this report will focus on the advancements of gate-controlled quantum dots exclusively. The most matured spin or charge state encoding of QDs has been achieved using GaAs/AlGaAs heterostructures, where the coherence time is limited by fluctuations in environment. Therefore, fabrication of QDs using materials which have a low abundance of nuclear spin, such as carbon, C and silicon, Si reduces the spin state decoherence route [\citen{Schroer2008TimeOut}]. This report reviews recent developments of quantum dot qubit implementation where the DiVincenzo criteria [\citen{Divincenzo2000TheComputation}] provides a means to assess the future prospects of QDs satisfying the requirements to realise a quantum computer.


